\href{http://mazenafif80@gmail.com}{}Biotechnology for production of silk.
Fermentation of red fruits of Withania Somnifera (Ashwagandah), in the production of silk .
Seen wolf spiders near the fermentation vessel.
first: Add hot water to the yeast fungus, and leave it a minutes before adding the fruits. And store at room temp,and the light.Hole cover small opening to enter the oxygen .I found that it formed a thin sticky layer float above the fruits, and then began to grow white yarn, and increased intensity until the sixth day. 
"Silk fiber begin to form within the spider's silk gland from and acidified liquid crystalline- form solution ( protein concentration of 30 to 50 % ) using very small forces followed by drawing in the air after the fiber has left the spider's body1.On this , we can say that spiders eat the fruits of plant and then the bacteria which converts to the silk protein, and adding enzymes to accelerate interaction . So seen the wolf spider around the continer. 
Properties: 
Threaded begin to appear in the steam droplets condensed on the wall's of the container.
It is not affected by HCl.
It is not affected by Acetic acid.
It is affected by (Na2Co3), into a gel, and disintegrate weaved to single filament. 
Reaction will not happen if stored in the dark.
If exposed to the air , changes color.
These fruits contain ( Polysacharides), so as to contain "Starch" and therefore scarce water solubility and slimy solution consists.

\end{itemize}}